% Standartpackete=================================
\documentclass[ngerman, 		% neue deutsche Rechtschr.
				12pt,			% Standardschriftgroesse
				a4paper,		% Papierformat
				listof=totoc,	% Listen im Tabel of content anzeigen
				parskip=half,	%vertikale Absatztrennung
				]{scrreprt}
				
\usepackage[utf8]{inputenc}		% Zeichenkodierung utf8 für Umlaute etc.
\usepackage[T1]{fontenc}		% Schriftkodierung
\usepackage{babel}				% Sprachspezifische Anpassungen (Übersetzung und Silbentrennung)
\usepackage{lmodern}			% verbesserte Computer Modern Schrift
\usepackage{libertine}			% Linux Libertine Schrift
\usepackage{microtype}

\usepackage{csquotes}			%Deutsche Anführungszeichen
\usepackage[onehalfspacing]{setspace}% 1,5-facher Zeilabstand
\usepackage[justification=RaggedRight, singlelinecheck=false]{caption}			% Nutzung von \caption ausserhalb von Gleitumgebungen u. Manipulation caption, Captions sind linksbündig

% Seitenränder==================================
\usepackage[left=2cm, right=2cm,top=2.5cm, bottom=2.5cm]{geometry}

			


% Kopf und Fusszeile/Seitenzahlen========================
\usepackage[]{scrlayer-scrpage}	%Nachfolger von scrpage2 ,headsepline aktiviert linie unter kopfzeile
\pagestyle{scrheadings}

\ihead{\headmark}
\automark{section}	%teilt \headmark mit was es eintragen soll
\chead{}			%erzwingt dass chead leer ist, kp y
\cfoot{\pagemark}



%Grafiken================================================
\usepackage{graphicx,wrapfig}
\setkomafont{captionlabel}{\bfseries}				%caption für Bildbeschriftung wird fett geschriebn
\renewcaptionname{ngerman}{\figurename}{Abb.}		%ersetzt Abbilgung durch Abb. bei der Funktion \caption
\usepackage{tikz}

%Tabellen==========================================
\usepackage{xcolor, colortbl}
\definecolor{Gray}{gray}{0.85}
\newcolumntype{g}{>{\columncolor{Gray}}c}		%grau hinterlegte Spalte; zentriert

\usepackage{tabularx}
\renewcaptionname{ngerman}{\tablename}{Tab.}

%Mathe===============================================
\usepackage{amsmath}

%Einheiten==========================================
\usepackage{siunitx}
\sisetup{locale=DE}

%Formelzeichen=======================================


% Referenzieren=====================================
\usepackage[plainpages=false]{hyperref}			% Immer als letztes laden, für klickbare Hyperlinks
												% plainpages lässt zw. 2 und II unterscheiden
								% wenn cleveref benutzt wird, dann NACH hyperef laden (einzige Ausnahme)
\usepackage{cleveref}			% einfacher referenzieren (mit \cref{})
\hypersetup{colorlinks=true, allcolors=black}

%Literatur/Zitate===================================
\usepackage[backend=biber,		% biber für die Sortierung etc
			style=numeric-comp,	% numerischer Zitationstil in Zitation und Literaturverzeichnis
			sorting=none,		% keine Sortierung (chronologisch)
			bibwarn=true,		% Anzeige von Warnungen (einfacheres debugging)
			bibencoding=utf8,	% bib auch in utf8 Eigabekodierung
			]{biblatex}			% Erzeugen des Literaturverzeichnisses
\addbibresource{refs.bib}		% im Latex-Projektordner muss eine Datei refs.bib vorhanden sein

\begin{document}
	% Vorderteil
	\begin{titlepage}
	\renewcommand*{\thepage}{Titel}	%nötig damit hyperref die titelseite nicht mit 1 oder röm 1 verwechselt
	\centering
	%{ \LARGE \bfseries Fachhochschule Kiel}
	
	%{\Large University of Applied Siences}
	\begin{figure}
		\centering
		\includegraphics[keepaspectratio=true,scale=1.5]{pics/FH_Kiel_Logo_deut_rgb.jpg}
	\end{figure}
	
	\bigskip
	Fachbereich Maschinenwesen
	
	Maschinenbau
	
	\vspace{2cm}
	{\bfseries \LARGE Bericht}
	
	\vspace{1cm}
	{\LARGE \bfseries
		\rule{\linewidth}{1pt}\\	% \\ generiert einen Zeilenumbruch, wie eine Leerzeile.
		%usepackage{csquotes} nötig für \enquote{text}
		"Titel der Arbeit"\\
		\rule{\linewidth}{1pt}
	}
	
	\vspace{1cm}
	vorgelegt von\\
	\vspace{0.5cm}
	Daniel Mansfeldt, XXXXXXXX\\
	\vspace{8cm}
	\raggedleft{Kiel, den \today}
\end{titlepage}

	\pagenumbering{Roman}
	\tableofcontents
	\clearpage
	
	% Haupteil
	\pagenumbering{arabic}
	\chapter{Einleitung}\label{ch:Einleitung}
hi
	 \chapter{Knotenverschiebung}\label{ch:Knotenverschiebung}
Zunächst soll bestimmt werden, welche Knotenverschiebungen die Belastungen hervorrufen.
\section{Grundlagen}\label{sec:Grundlagen}
Für Stabwerke gilt der folgende Zusammenhang \cite{Moldenhauer.2018}.
 
\begin{equation}\label{eq:Ku=F}
	K\vec{u}=\vec{F}
\end{equation}

Es liegt also ein Gleichungsystem in Matrixschreibweise vor. Dabei ist $\K$ die Gesamtsteifigkeitsmatrix des Stabwerks, $\vec{F}$ der Kraftvektor und $ \vec{u} $ der Schiebungsvektor. Der Kraftvektor $\vec{F}$  enthält alle Kräfte, die je Knoten in Richtung der Abzisse bzw. der Ordinate verlaufen. Die ersten beiden Komponenten von $\vec{F}$ sind also die Kräfte am Knoten 1 (erste Komponente in x-Richtung, zweite in y-Richtung). Entsprechend aufgebaut ist auch der Verschiebungsvektor $ \vec{u} $.

Damit $ \vec{u} $ bestimmt werden kann, sind also zunächst $ \K $ und $ \vec{F} $ zu bestimmen. 

Mit der Erzeugung sämtlicher Stab-Objekte sind all deren Element-Steifigkeitsmatrizen in globalen Koordinaten und deren Koinzidenzmatrizen bekannt. Damit kann die Gesamtsteifigkeitsmatrix des Stabwerks bestimmt werden. Sie bildet die Summe aller Element-Steifigkeitsmatrizen multipliziert mit den Koinzidenzmatrizen des Stab und deren Transponierten \cite{Moldenhauer.2018}.
\begin{equation}\label{eq:Gesamtsteifigkeitsmatrix}
	\K=\sum_i\I_i^T\K_i\I_i
\end{equation}

Da das Stabwerk gelagert ist, unterliegt es Randbedingung bezüglich der Verschiebungen. So sind $ u_1=v_1=v_4=0 $, da sich am Knoten 1 ein Festlager befindet und an Knoten 4 ein Loslager, welches Verschiebungen in vertikaler Richtung sperrt. Um die Randbedingungen zu berücksichtigen wird das Gleichungssystem aus \ref{eq:Ku=F} reduziert. D.h. es werden, entsprechend der Positionen der Randbedingungen, Zeilen und Spalten aus dem Gleichungssystem gestrichen. Erst das reduzierte Gleichungssystem kann mittels \textit{Gauß-Jordan} gelöst werden, da die noch unbekannten Lagerkräfte nur in je einer Gleichung auftauchen. Sie können also nicht durch das Eliminationsverfahren behandelt werden. Folglich kann nur das reduzierte Gleichungssystem in die "`Reduced Row Ecolon"'-Form gebracht werden, was die Voraussetzung zur Lösung nach \textit{Gauß-Jordan} darstellt.

\section{Umsetzung in Python}
Um die Verschiebungen vom Rechner bestimmen zu lassen werden zunächst die Gesamtsteifigkeitsmatrix bestimmt. Diese und der Kraftvektor werden anschließend, den Randbegingungen entsprechend, reduziert. Dazu wird eine Liste mit Indizes definiert, die die Zeilen und Spalten beschreiben, die aus den Matrizen bzw. Vektoren zu entfernen sind.
\subsection{Gesamtsteifigkeitsmatrix}\label{subsec:Gesamtsteifigkeitsmatrix}
Um die Gesamtsteifigkeitsmatrix $ \K $ zu bestimmen, wird \ref{eq:Gesamtsteifigkeitsmatrix} in Python umgesetzt. Mittels einer \textit{for}-Schleife wird über alle Stäbe (Werte des Dictionaries "`stab"') iteriert. So können die Attribute $ \K_{Stab} $ und $ \I_{Stab} $ aller Stab-Objekte abgerufen werden und zur Gesamtsteifigkeitsmatrix $ \K $ aufsummiert werden. Im Anschluss wird $ \K $, den Randbedingungen entsprechend, reduziert.

\subsection{Kraftvektor}\label{subsec:Kraftvektor}
Der Kraftvektor $ \vec{F} $ wird manuell definiert. Da die Lagerkräfte unbekannt sind, wird als Platzhalter ein \textit{None}-Objekt an deren Stelle verwendet. Eben diese Komponenten werden anschießend aus dem Kraftvektor zur Berücksichtigung der Randbedingungen entfernt.

\subsection{Ergebnisse}\label{subsec:Ergebnisse Aufgabe 1}
Mit \ref{eq:Ku=F} bleibt damit ein Gleichungsystem für die unbekannten verschiebungen übrig. Dieses wird nach \textit{Gauß-Jordan} gelöst. Die Verschiebungen liegen alle im Bereich einstelliger Millimeterwerte. Für ein reales Stabwerk, dieser Größenordnung sind dies plausible Ergebnisse.
	\chapter{Teil 2: Dichte}\label{ch:Dichte}
Um die Masse des Stabwerks bestimmen zu können, muss zunächst die Dichte des verwendeten Werkstoffs festliegen. Es wird davon ausgegangen, dass alle Stäbe aus dem gleichen homogenen Material bestehen.

Es stehen 1000 Messwerte der Materialdichte zur Verfügung. Bei dieser Menge an Daten wird angenommen, dass sie gut die tatsächlichen Dichtewerte im realen Stab abbilden. Die Annahme vom homogenen Material verträgt sich daher mit der Festlegung eines konservativen Dichtewertes. 

Da aus einer realen Menge an Werten nicht der wahr Mittelwert, bestimmt werden kann, dienen sog. Konfidenzintervalle dazu den Bereich einzugrenzen in dem der wahre Mittelwert mit einer gewissen Wahrscheinlichkeit liegt. Für den Bereich $ \pm2\cdot SE $ (Standarderror) um den Mittelwert der Daten, beträgt die Aufenthaltswahrscheinlichkeit für den wahren Mittelwert $ \mu $ 95\%. Diese Wahrscheinlichkeit wird als ausreichend hoch angesehen. Als Annahme auf der sicheren Seite (im Sinne konservativer Berechnung) wird als Dichte derjenige Wert gewählt, der dem oberen Rand des Konfidenzintervalls entspricht. Die Wahrscheinlichkeit, dass der wahre Mittelwert niedriger ist als dieser ist mit 97,5\% ausreichend groß, sodass eine Unterschätzung des Mittelwerts als unwahrscheinlich gewertet werden kann. Das setzt selbstverständlich zuverlässige Messwerte voraus.
	\include{docs/Teil3}
	
	% Schlussteil
	\printbibliography[heading=bibintoc]
	\listoffigures
	\listoftables
\end{document}

Aufbau:

Einleitung
	- Was ist die Aufgabe?
	- Darstellung des STabwerks
	- Modelbeschreibung: Wie haben wir das STabwerk in pyhton umgesetzt?
	
Teil 1:
	- FEM Theorie zur Berechnung von Verschiebungen
	- Beschreibung des Gauß-Jordan Lösungsverfahren
	- Umsetzung in Python (Methoden und Klassenattribute)
	- Bewertung der Ergebnisse
	
Teil 2:
	- Interpretation der Aufgabenstellung
	- Begründung/Beurteilung des Lösungsvorgehens. Warum ist das gewählte Verfahren zur Bestimmung der Dichte gut geeignet?
	- Berechnung des Ergebnisses mit Unsetzung in python
	- Beurteilung/Einschätzung des ERgebnisses.
	
Teil 3:
	- Problembeschreibung: Warum soll nur der Querschnitt variiert werden um die Masse zu minimieren? 
	- Beschreibung und Beurteilung des Vorgehens
	- Umsetzung in Python und berechnung
	- Bewertung der Ergebnisse
	