% Standartpackete=================================
\documentclass[ngerman, 		% neue deutsche Rechtschr.
				12pt,			% Standardschriftgroesse
				a4paper,		% Papierformat
				listof=totoc,	% Listen im Tabel of content anzeigen
				parskip=half,	%vertikale Absatztrennung
				]{scrreprt}
				
\usepackage[utf8]{inputenc}		% Zeichenkodierung utf8 für Umlaute etc.
\usepackage[T1]{fontenc}		% Schriftkodierung
\usepackage{babel}				% Sprachspezifische Anpassungen (Übersetzung und Silbentrennung)
\usepackage{lmodern}			% verbesserte Computer Modern Schrift
\usepackage{libertine}			% Linux Libertine Schrift
\usepackage{microtype}

\usepackage{csquotes}			%Deutsche Anführungszeichen
\usepackage[onehalfspacing]{setspace}% 1,5-facher Zeilabstand
\usepackage[justification=RaggedRight, singlelinecheck=false]{caption}			% Nutzung von \caption ausserhalb von Gleitumgebungen u. Manipulation caption, Captions sind linksbündig

% Seitenränder==================================
\usepackage[left=2cm, right=2cm,top=2.5cm, bottom=2.5cm]{geometry}

			


% Kopf und Fusszeile/Seitenzahlen========================
\usepackage[]{scrlayer-scrpage}	%Nachfolger von scrpage2 ,headsepline aktiviert linie unter kopfzeile
\pagestyle{scrheadings}

\ihead{\headmark}
\automark{section}	%teilt \headmark mit was es eintragen soll
\chead{}			%erzwingt dass chead leer ist, kp y
\cfoot{\pagemark}



%Grafiken================================================
\usepackage{graphicx,wrapfig}
\setkomafont{captionlabel}{\bfseries}				%caption für Bildbeschriftung wird fett geschriebn
\renewcaptionname{ngerman}{\figurename}{Abb.}		%ersetzt Abbilgung durch Abb. bei der Funktion \caption
\usepackage{tikz}

%Tabellen==========================================
\usepackage{xcolor, colortbl}
\definecolor{Gray}{gray}{0.85}
\newcolumntype{g}{>{\columncolor{Gray}}c}		%grau hinterlegte Spalte; zentriert

\usepackage{tabularx}
\renewcaptionname{ngerman}{\tablename}{Tab.}

%Mathe===============================================
\usepackage{amsmath}

%Einheiten==========================================
\usepackage{siunitx}
\sisetup{locale=DE}

%Formelzeichen=======================================


% Referenzieren=====================================
\usepackage[plainpages=false]{hyperref}			% Immer als letztes laden, für klickbare Hyperlinks
												% plainpages lässt zw. 2 und II unterscheiden
								% wenn cleveref benutzt wird, dann NACH hyperef laden (einzige Ausnahme)
\usepackage{cleveref}			% einfacher referenzieren (mit \cref{})
\hypersetup{colorlinks=true, allcolors=black}

%Literatur/Zitate===================================
\usepackage[backend=biber,		% biber für die Sortierung etc
			style=numeric-comp,	% numerischer Zitationstil in Zitation und Literaturverzeichnis
			sorting=none,		% keine Sortierung (chronologisch)
			bibwarn=true,		% Anzeige von Warnungen (einfacheres debugging)
			bibencoding=utf8,	% bib auch in utf8 Eigabekodierung
			]{biblatex}			% Erzeugen des Literaturverzeichnisses
\addbibresource{refs.bib}		% im Latex-Projektordner muss eine Datei refs.bib vorhanden sein