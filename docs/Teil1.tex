 \chapter{Knotenverschiebung}\label{ch:Knotenverschiebung}
Zunächst soll bestimmt werden, welche Knotenverschiebungen die Belastungen hervorrufen.
\section{Grundlagen}\label{sec:Grundlagen FEM}
Für Stabwerke gilt der folgende Zusammenhang \cite{Moldenhauer.2018}.
 
\begin{equation}\label{eq:Ku=F}
	K\vec{u}=\vec{F}
\end{equation}

Es liegt also ein Gleichungsystem in Matrixschreibweise vor, welches die Belastung der Knoten eines Stabwerks mit den Verschiebungen der Knoten in Zusammenhang stellt. Ausgedrückt wird dieser Zusammenhang durch die Gesamtsteifigkeitsmatrix $ \K $. In \ref{eq:Ku=F} ist $\vec{F}$ der Kraftvektor und $ \vec{u} $ der Verschiebungsvektor.

Der Kraftvektor $\vec{F}$  enthält alle Kräfte, die je Knoten in Richtung der Abzisse bzw. der Ordinate verlaufen. Die ersten beiden Komponenten von $\vec{F}$ sind also die Kräfte am Knoten 1 (erste Komponente in x-Richtung, zweite in y-Richtung). Entsprechend aufgebaut ist auch der Verschiebungsvektor $ \vec{u} $. Allgemein haben $ \vec{F} $ und $ \vec{u} $ also folgende Formen:

\begin{equation*}
	\begin{pmatrix}
	\text{Knoten 1}_x\\
	\text{Knoten 1}_y\\
	\text{Knoten 2}_x\\
	\text{Knoten 2}_y\\
	\vdots
	\end{pmatrix}
\end{equation*}
Die Indizes stehen dabei für die Knotennummer bzw. die Koordinatenrichtung. Im Verschiebungsvektor werden Verschiebungen in x-Richtung mit "`u"' und solche in y-Richtung mit "`v"' abgekürzt.

Damit $ \vec{u} $ bestimmt werden kann, sind also zunächst $ \K $ und $ \vec{F} $ zu bestimmen. 

Sind alle Element-Steifigkeitsmatrizen der Stäbe (in globalen Koordinaten) und deren Koinzidenzmatrizen bekannt, kann die Gesamtsteifigkeitsmatrix des Stabwerks bestimmt werden. Sie bildet die Summe aller Element-Steifigkeitsmatrizen multipliziert mit den Koinzidenzmatrizen des Stabs und deren Transponierten \cite{Moldenhauer.2018}.
\begin{equation}\label{eq:Gesamtsteifigkeitsmatrix}
	\K=\sum_i\I_i^T\K_i\I_i
\end{equation}

Da das Stabwerk gelagert ist, unterliegt es Randbedingung bezüglich der Verschiebungen. So sind $ u_1=v_1=v_4=0 $, da sich am Knoten 1 ein Festlager befindet und an Knoten 4 ein Loslager befindet. Um die Randbedingungen zu berücksichtigen wird das Gleichungssystem aus \ref{eq:Ku=F} reduziert. D.h. es werden, entsprechend der Positionen der Randbedingungen, Zeilen und Spalten aus dem Gleichungssystem gestrichen. Erst das reduzierte Gleichungssystem kann mittel dem Lösungsverfahren nach \textit{Gauß-Jordan} gelöst werden, da die noch unbekannten Lagerkräfte nur in je einer Gleichung auftauchen. Sie können also nicht durch das Eliminationsverfahren behandelt werden. Folglich kann nur das reduzierte Gleichungssystem in die "`Reduced Row Ecolon"'-Form gebracht werden, was die Voraussetzung zur Lösung nach \textit{Gauß-Jordan} darstellt.

\section{Umsetzung in Python}

\subsection{Gesamtsteifigkeitsmatrix}\label{subsec:Gesamtsteifigkeitsmatrix}
Um die Gesamtsteifigkeitsmatrix $ \K $ zu bestimmen und diese der Stabwerks-Instanz als Attribut zuzuweisen, wird die Methode "`calc\_K"' in der Klasse "`Stabwerk"' definiert. Darin wird mittels einer \textit{for}-Schleife wird über alle Objekte im Attribut "`Staebe"' iteriert. So können die Attribute $ \K_{Stab} $ und $ \I_{Stab} $ aller Stab-Objekte abgerufen werden und nach \ref{eq:Gesamtsteifigkeitsmatrix} zur Gesamtsteifigkeitsmatrix $ \K $ aufsummiert werden. Durch eine weitere Methode wird $ \K $, den Randbedingungen entsprechend, reduziert.

\subsection{Berechnung der Verschiebungen}\label{subsec:Berechnung der Verschiebungen}
Wie in Kapitel \ref{sec:Grundlagen} beschrieben werden nur die unbekannten Knotenverschiebungen berechnet. Nach Gauß-Jordan löst die Inverse der reduzierten Steifigkeitsmatrix $ \K $, multipliziert mit dem Kraftvektor, das Gleichungssystem \ref{eq:Ku=F} für die gesuchten Verschiebungen auf.

Um den vollständigen Verschiebungsvektor ausgeben zu können, werden die Verschiebungen als den Lagerbedingungen jetzt in den Ergebnissvektor eingesetzt.
Der vollständige Verschiebungsvektor lautet
\begin{equation*}
	\vec{u}=\begin{pmatrix}
		0.0 \\ 
		0.0 \\ 
		0.0\\ 
		0.0\\ 
		0.0\\ 
		
	\end{pmatrix} 
\end{equation*}

\subsection{Ergebnisse}\label{subsec:Ergebnisse Aufgabe 1}
Mit \ref{eq:Ku=F} bleibt damit ein Gleichungsystem für die unbekannten verschiebungen übrig. Dieses wird nach \textit{Gauß-Jordan} gelöst. Die Verschiebungen liegen alle im Bereich einstelliger Millimeterwerte. Für ein reales Stabwerk, dieser Größenordnung sind dies plausible Ergebnisse.