\chapter{Knotenverschiebung}\label{ch:Knotenverschiebung}
Zunächst soll bestimmt werden, welche Knotenverschiebungen die Belastungen hervorrufen.
\section{Grundlagen}\label{sec:Grundlagen}
Für Stabwerke gilt der folgende Zusammenhang.
 
\begin{equation}\label{eq:Ku=F}
	K\vec{u}=\vec{F}
\end{equation}

Dabei ist $\K$ die Gesamtsteifigkeitsmatrix des Stabwerks, $\vec{F}$ der Kraftvektor und $ \vec{u} $ der Schiebungsvektor. Der Kraftvektor ist enthält alle Kräfte, die je Knoten in Richtung der Abzisse bzw. der Ordinate verlaufen.

$ \K $ berechnet sich aus den lokalen Steifigkeitsmatrizen der Stäbe, deren Orienteirung im globalen Koordinatensystem (ausgedrückt durch den Winkel $ \alpha $) und deren Koinzidenzmatrix, die eine Zuordnung der lokalen zu den globalen Knoten beschreibt.

Der Verschiebungsvektor $ \vec{u} $ enthält sämtliche Verschiebungen aller Knoten. Die Verschiebungen sind entsprechend ihrer Richtungsanteile aufgeteilt.

\section{Problembeschreibung}\label{sec:Problembeschreibung}
Um die Verschiebungen aller Knoten zu bestimmen, die nicht durch die Definition eines Lagers beschrieben werden, müssen zunächst $ \K $ und $ \vec{F} $ ermittelt werden.

\subsection{Globale Steifigkeitsmatrix}\label{subsec:Globale Steifigkeitsmatrix}
Die globale Steifigkeitsmatrix $ \K $ des Stabwerks wird in drei Schritten bestimmt. Zunächst wird mit Hilfe der allgemeinen globalen Steifigkeitsmatrix der Stäbe $ \K_{Stab} $ die Steifigkeitsmatrizen jedes einzelnen Stabs in globalen Koordinaten bestimmt. Dazu sind der E-Modul $ E $, die Querschnittsfläche $ A $, die Länge $ l $ und der Winkel $ \alpha $ um den lokalen Knoten 1 in $ \K_{Stab} $ einzusetzen.!!!!!Zitat Moldenhauer!!!!!!!!

Im zweiten Schritt wird die Koinzidenzmatrix $ \I_{Stab} $ für jeden der Stäbe aufgestellt. Eine Koinzidenztabelle gemäß Moldenahueer wurde dazu in Python umgesetzt.

Für den letzten Schritt wird die Steifigkeitsmatrix des Stabwerks wie folgt bestimmt !!!!zitat moldenhauer!!!!!!.
\begin{equation}\label{eq:Gesamtsteifigkeitsmatrix}
\K=\sum_i\I_i^T\K_i\I_i
\end{equation}

\subsection{Kraftvektor}\label{subsec:Kraftvektor}
Die Komponenten des Kraftvektors $ \vec{F} $ sind größtenteils gegebene Größen. Sie werden entsprechend ihrer Richtung und dem Knoten an dem sie angreifen in den Kraftvektor eingetragen. Lediglich die Auflagerreaktionen sind nicht bekannt. 

\section{Aufbereitung des Gleichungssystems}\label{sec:Aufbereitung des GL}
Um die Verschiebungen trotz der unbekannten Lagerreaktionen bestimmen zu können, muss das Gleichungssystem reduziert werden. Zu diesem Zweck wird $ \vec{u} $ in die Vektoren $ \vec{u}_b $ und $ \vec{u}_g $ aufgeteilt, sodass gilt
\begin{equation}\label{eq:u=ub+ug}
	\vec{u}=\vec{u}_b+\vec{u}_g
\end{equation}
$ \vec{u}_b $ enthält dabei alle vorgegebenen Verschiebungen und den Wert 0 an den Stellen die gesuchten Verschiebungen entsprechen. $ \vec{u}_g $ trägt folglich nur die gesuchten Verschiebungen und Nullen für die die durch die Lagerungen bekannt sind.

Da alle bekannten Verschiebungen aus der Lagerung stammen und diese null sind, ist $ \vec{u}_b=\vec{0} $. Für die weitere Berechnung ist $ \vec{u}_b $ also nicht relevant.

Um das Gleichungssystem auf die gleiche Anzahl Gleichungen zu reduzieren wie die Anzahl gesuchter Verschiebungen, werden die Zeilen und Spalten der Matrixschreibweise gestrichen, in denen der Vektor der gesuchten Verschiebungen $ \vec{u}_g $ den Wert null hat. Das so reduzierte Gleichungssystem kann jetzt gelöst werden. 