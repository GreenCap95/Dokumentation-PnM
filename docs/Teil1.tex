 \chapter{Knotenverschiebung}\label{ch:Knotenverschiebung}
Aufgabenteil 1 ist es, die Verschiebungen der Knoten zu bestimmen, die durch die Belastung $\vec{F}$ hervorgerufen werden. (s. \cref{ch:Einleitung})
\section{Grundlagen}\label{sec:Grundlagen FEM}
Für Stabwerke gilt nach \cite{Moldenhauer.2018} der folgende Zusammenhang.
 
\begin{equation}\label{eq:Ku=F}
	K\vec{u}=\vec{F}
\end{equation}

Es liegt also ein Gleichungssystem in Matrixschreibweise vor, welches die Belastung eines Stabwerks mit den Verschiebungen der Knoten in Zusammenhang stellt. Ausgedrückt wird dieser Zusammenhang durch die Gesamtsteifigkeitsmatrix $ \K $. In \ref{eq:Ku=F} ist $\vec{F}$ der Kraftvektor und $ \vec{u} $ der Verschiebungsvektor.

Der Kraftvektor $\vec{F}$  enthält je Knoten zwei Kraft-Komponenten. Je eine pro Koordinaten-Richtung. Die ersten beiden Komponenten von $\vec{F}$ sind also die Kräfte am Knoten 1 (erste Komponente in x-Richtung, zweite in y-Richtung). Entsprechend aufgebaut ist auch der Verschiebungsvektor $ \vec{u} $. Allgemein haben $ \vec{F} $ und $ \vec{u} $ also folgende Formen:

\begin{equation*}
	\begin{pmatrix}
	\text{Knoten 1}_x\\
	\text{Knoten 1}_y\\
	\text{Knoten 2}_x\\
	\text{Knoten 2}_y\\
	\vdots
	\end{pmatrix}
\end{equation*}
Im Verschiebungsvektor werden Verschiebungen in x-Richtung mit "`u"' und solche in y-Richtung mit "`v"' abgekürzt.

Damit $ \vec{u} $ bestimmt werden kann, sind zunächst $ \K $ und $ \vec{F} $ zu bestimmen. 

Sind alle Element-Steifigkeitsmatrizen der Stäbe (in globalen Koordinaten) und deren Koinzidenzmatrizen bekannt, kann die Gesamtsteifigkeitsmatrix des Stabwerks bestimmt werden. Sie bildet sich aus der Summe aller Element-Steifigkeitsmatrizen multipliziert mit den Koinzidenzmatrizen des Stabs und deren Transponierten \cite{Moldenhauer.2018}.
\begin{equation}\label{eq:Gesamtsteifigkeitsmatrix}
	\K=\sum_i\I_i^T\K_i\I_i
\end{equation}

Da das Stabwerk gelagert ist, unterliegt es Randbedingung bezüglich der Verschiebungen. So sind $ u_1=v_1=v_4=0 $, da sich am Knoten 1 ein Festlager befindet und an Knoten 4 ein Loslager befindet. Um die Randbedingungen zu berücksichtigen wird das Gleichungssystem aus \ref{eq:Ku=F} reduziert. D.h. es werden, entsprechend der Positionen der Randbedingungen, Zeilen und Spalten aus dem Gleichungssystem gestrichen. Erst das reduzierte Gleichungssystem kann mittel dem Lösungsverfahren nach \textit{Gauß-Jordan} gelöst werden, da die noch unbekannten Lagerkräfte nur in je einer Gleichung auftauchen. Sie können also nicht durch das Eliminationsverfahren behandelt werden. Folglich kann nur das reduzierte Gleichungssystem in die "`Reduced Row Ecolon"'-Form gebracht werden, was die Voraussetzung zur Lösung nach \textit{Gauß-Jordan} darstellt.

\section{Lösungsverfahren nach Gauß-Jordan}
Ein lineares Gleichungssystem (LGS) kann in Matrix-Schreibweise ausgedrückt werden. Seine allgemeine Form ist dann
\begin{equation}\label{eq:lgs matrixschreibweise}
	Ax=b
\end{equation}

Ein gängiges Verfahren zur Lösung linearer Gleichungssysteme ist das Gauß´sche Eliminationsverfahren. Dabei werden durch geschicktes Zusammenführen je zweier Gleichungen der Reihe nach alle Unbekannten bestimmt. Das Prinzip dieses händischen Vorgehens kann auf die Matrix-Schreibweise übertragen und damit vom Rechner angewendet werden.

Überführt man $ A $ aus \cref{eq:lgs matrixschreibweise} durch Elimination in die Einheitsmatrix, so lassen sich die gesuchten Unbekannten ablesen, da links lediglich die Unbekannten (Komponenten von $ x $) verbleiben. Man spricht von einer "`Reduced Row Ecolon"'-Form des LGS. Da für die Einheitsmatrix
\begin{equation}\label{eq:Definition Inversematrix}
	E=A^{-1}A
\end{equation}
gilt, folgt aus \cref{eq:lgs matrixschreibweise}
\begin{equation}\label{eq:Lösung Gauß-Jordan}
	x=A^{-1}b
\end{equation}

Angewendet auf \cref{eq:Ku=F} bedeutet dies also, dass die gesuchten Verschiebungen bestimmt werden können indem man die Inverse der Gesamtsteifigkeitsmatrix $ \K $ mit dem Kraftvektor $ \vec{F} $ multipliziert.

\section{Umsetzung in Pyhton}\label{sec:Umsetzung in Pyhton}

\subsection{Gesamtsteifigkeitsmatrix}\label{subsec:Gesamtsteifigkeitsmatrix}
Um die Gesamtsteifigkeitsmatrix $ \K $ zu bestimmen und diese der Stabwerks-Instanz als Attribut zuzuweisen, wird die Methode "`calc\_K"' definiert. Darin wird mittels einer \textit{for}-Schleife wird über alle Stäbe iteriert. So können die Attribute $ \K_{Stab} $ und $ \I_{Stab} $ aller Stab-Objekte abgerufen und nach \ref{eq:Gesamtsteifigkeitsmatrix} zur Gesamtsteifigkeitsmatrix $ \K $ aufsummiert werden. Durch eine weitere Methode "`reduce\_K"' wird $ \K $, den Randbedingungen entsprechend, reduziert.

\section{Berechnung der Verschiebungen}\label{subsec:Berechnung der Verschiebungen}
Das Python-Modul \textit{numpy} bietet implementierte Methoden zur Bestimmung der Inversen einer Matrix, sowie zur Matrix-Multiplikation. Damit lassen sich die gesuchten Verschiebungen direkt und, im Rahmen der Rechnergenauigkeit, exakt bestimmen.

Wie in Kapitel \ref{sec:Grundlagen} beschrieben werden nur die unbekannten Knotenverschiebungen berechnet. Das Lösungsverfahren nach Gauß-Jordan beruht auf der Gauß-Elemination. D.h. durch geschicktes Zusammenführen je zweier Gleichungen lässt sich das lineare Gleichungssystem für eine Unbekannte nach der anderen auflösen. Angewendet auf eine Matrix-Schreibweise bedeutet dies, dass das Gleichungsystem in eine "`Reduced Row Ecolon"'-Form überführt wird. löst die Inverse der reduzierten Steifigkeitsmatrix $ \K $, multipliziert mit dem Kraftvektor, das Gleichungssystem \ref{eq:Ku=F} für die gesuchten Verschiebungen auf.

Um den vollständigen Verschiebungsvektor ausgeben zu können, werden die Verschiebungs der Lagerbedingungen an den entsprechenden Stellen (Indizes s. \cref{subsubsec:Attribute-Stabklasse}) eingesetzt.
Der vollständige Verschiebungsvektor lautet
\begin{equation*}
	\vec{u}=\begin{pmatrix}
		0.0 \\ 
		0.0 \\ 
		0.0\\ 
		0.0\\ 
		0.0\\ 
		
	\end{pmatrix} 
\end{equation*}