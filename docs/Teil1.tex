 \chapter{Knotenverschiebung}\label{ch:Knotenverschiebung}
Zunächst soll bestimmt werden, welche Knotenverschiebungen die Belastungen hervorrufen.
\section{Grundlagen}\label{sec:Grundlagen}
Für Stabwerke gilt der folgende Zusammenhang \cite{Moldenhauer.2018}.
 
\begin{equation}\label{eq:Ku=F}
	K\vec{u}=\vec{F}
\end{equation}

Es liegt also ein Gleichungsystem in Matrixschreibweise vor. Dabei ist $\K$ die Gesamtsteifigkeitsmatrix des Stabwerks, $\vec{F}$ der Kraftvektor und $ \vec{u} $ der Schiebungsvektor. Der Kraftvektor $\vec{F}$  enthält alle Kräfte, die je Knoten in Richtung der Abzisse bzw. der Ordinate verlaufen. Die ersten beiden Komponenten von $\vec{F}$ sind also die Kräfte am Knoten 1 (erste Komponente in x-Richtung, zweite in y-Richtung). Entsprechend aufgebaut ist auch der Verschiebungsvektor $ \vec{u} $.

Damit $ \vec{u} $ bestimmt werden kann, sind also zunächst $ \K $ und $ \vec{F} $ zu bestimmen. 

Mit der Erzeugung sämtlicher Stab-Objekte sind all deren Element-Steifigkeitsmatrizen in globalen Koordinaten und deren Koinzidenzmatrizen bekannt. Damit kann die Gesamtsteifigkeitsmatrix des Stabwerks bestimmt werden. Sie bildet die Summe aller Element-Steifigkeitsmatrizen multipliziert mit den Koinzidenzmatrizen des Stab und deren Transponierten \cite{Moldenhauer.2018}.
\begin{equation}\label{eq:Gesamtsteifigkeitsmatrix}
	\K=\sum_i\I_i^T\K_i\I_i
\end{equation}

Da das Stabwerk gelagert ist, unterliegt es Randbedingung bezüglich der Verschiebungen. So sind $ u_1=v_1=v_4=0 $, da sich am Knoten 1 ein Festlager befindet und an Knoten 4 ein Loslager, welches Verschiebungen in vertikaler Richtung sperrt. Um die Randbedingungen zu berücksichtigen wird das Gleichungssystem aus \ref{eq:Ku=F} reduziert. D.h. es werden, entsprechend der Positionen der Randbedingungen, Zeilen und Spalten aus dem Gleichungssystem gestrichen. Erst das reduzierte Gleichungssystem kann mittels \textit{Gauß-Jordan} gelöst werden, da die noch unbekannten Lagerkräfte nur in je einer Gleichung auftauchen. Sie können also nicht durch das Eliminationsverfahren behandelt werden. Folglich kann nur das reduzierte Gleichungssystem in die "`Reduced Row Ecolon"'-Form gebracht werden, was die Voraussetzung zur Lösung nach \textit{Gauß-Jordan} darstellt.

\section{Umsetzung in Python}
Um die Verschiebungen vom Rechner bestimmen zu lassen werden zunächst die Gesamtsteifigkeitsmatrix bestimmt. Diese und der Kraftvektor werden anschließend, den Randbegingungen entsprechend, reduziert. Dazu wird eine Liste mit Indizes definiert, die die Zeilen und Spalten beschreiben, die aus den Matrizen bzw. Vektoren zu entfernen sind.
\subsection{Gesamtsteifigkeitsmatrix}\label{subsec:Gesamtsteifigkeitsmatrix}
Um die Gesamtsteifigkeitsmatrix $ \K $ zu bestimmen, wird \ref{eq:Gesamtsteifigkeitsmatrix} in Python umgesetzt. Mittels einer \textit{for}-Schleife wird über alle Stäbe (Werte des Dictionaries "`stab"') iteriert. So können die Attribute $ \K_{Stab} $ und $ \I_{Stab} $ aller Stab-Objekte abgerufen werden und zur Gesamtsteifigkeitsmatrix $ \K $ aufsummiert werden. Im Anschluss wird $ \K $, den Randbedingungen entsprechend, reduziert.

\subsection{Kraftvektor}\label{subsec:Kraftvektor}
Der Kraftvektor $ \vec{F} $ wird manuell definiert. Da die Lagerkräfte unbekannt sind, wird als Platzhalter ein \textit{None}-Objekt an deren Stelle verwendet. Eben diese Komponenten werden anschießend aus dem Kraftvektor zur Berücksichtigung der Randbedingungen entfernt.

\subsection{Ergebnisse}\label{subsec:Ergebnisse Aufgabe 1}
Mit \ref{eq:Ku=F} bleibt damit ein Gleichungsystem für die unbekannten verschiebungen übrig. Dieses wird nach \textit{Gauß-Jordan} gelöst. Die Verschiebungen liegen alle im Bereich einstelliger Millimeterwerte. Für ein reales Stabwerk, dieser Größenordnung sind dies plausible Ergebnisse.