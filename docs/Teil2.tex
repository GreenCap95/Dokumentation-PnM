\chapter{Teil 2: Dichte}\label{ch:Dichte}
Um die Masse des Stabwerks bestimmen zu können, muss zunächst die Dichte des verwendeten Werkstoffs festliegen. Es wird davon ausgegangen, dass alle Stäbe aus dem gleichen homogenen Material bestehen.

Es stehen 1000 Messwerte der Materialdichte zur Verfügung. Bei dieser Menge an Daten wird angenommen, dass sie gut die tatsächlichen Dichtewerte im realen Stab abbilden. Die Annahme vom homogenen Material verträgt sich daher mit der Festlegung eines konservativen Dichtewertes. 

Da aus einer realen Menge an Werten nicht der wahr Mittelwert, bestimmt werden kann, dienen sog. Konfidenzintervalle dazu den Bereich einzugrenzen in dem der wahre Mittelwert mit einer gewissen Wahrscheinlichkeit liegt. Für den Bereich $ \pm2\cdot SE $ (Standarderror) um den Mittelwert der Daten, beträgt die Aufenthaltswahrscheinlichkeit für den wahren Mittelwert $ \mu $ 95\%. Diese Wahrscheinlichkeit wird als ausreichend hoch angesehen. Als Annahme auf der sicheren Seite (im Sinne konservativer Berechnung) wird als Dichte derjenige Wert gewählt, der dem oberen Rand des Konfidenzintervalls entspricht. Die Wahrscheinlichkeit, dass der wahre Mittelwert niedriger ist als dieser ist mit 97,5\% ausreichend groß, sodass eine Unterschätzung des Mittelwerts als unwahrscheinlich gewertet werden kann. Das setzt selbstverständlich zuverlässige Messwerte voraus.