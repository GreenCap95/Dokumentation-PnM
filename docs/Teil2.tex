\chapter{Teil 2: Dichte}\label{ch:Dichte}
Um die Masse des Stabwerks bestimmen zu können, muss zunächst die Dichte des verwendeten Werkstoffs festliegen. Es wird davon ausgegangen, dass alle Stäbe aus dem gleichen homogenen Material bestehen.

Um einen geeigneten Wert der Dichte zu bestimmen, steht eine Tabelle mit 1000 Werten zur Verfügung. Die Werte simulieren streuende Messwerte. Es gilt anhand dieser Daten auf einen Wert zu schließen, der die Dichte für das gesamte Stabwerk möglichst gut annähert.

Um einen ersten Eindruck über den gegebenen Datensatz zu erhalten, wird ein Histogramm erstellt. Die Daten scheinen eine Normalverteilung abzubilden. 